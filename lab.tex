% !TeX root 'lab.tex'
\documentclass{article}
\usepackage{amsmath}
\usepackage[T2A]{fontenc}
\usepackage[utf8]{inputenc}
\usepackage{icomma}
\usepackage{graphicx}
\usepackage{parcolumns}
\usepackage[a4paper, left=3cm, right=1.5cm, top=1.5cm, bottom=1.5cm]{geometry}


\begin{document}

\addtocounter{page}{-1}
\thispagestyle{empty}
\begin{figure}[h]
    \centering
    \def\svgwidth{0.33\textwidth}
    \input{Станкин.pdf_tex}
\end{figure}
\begin{center}
    \textbf{МИНОБРНАУКИ РОССИИ\\
        федеральное государственное автономное образовательное учреждение
        высшего образования
        «Московский государственный технологический университет «СТАНКИН»
        (ФГАОУ ВО «МГТУ «СТАНКИН»)}
\end{center}
\noindent\makebox[\linewidth]{\rule{\columnwidth}{0.4pt}}
\textbf{Институт \hfill Кафедра \\
    информационных \hfill информационных технологий \\
    технологий \hfill и вычислительных систем}
\\
\vfill
\begin{center}
    ОТЧЕТ О ВЫПОЛНЕНИИ\\
    ЛАБОРАТОРНОЙ РАБОТЫ ПО ДИСЦИПЛИНЕ\\
    «Вычислительная математика»
\end{center}
\vfill
СТУДЕНТА 2 КУРСА \hfill баклавариата \hfill ГРУППЫ ИДБ-23-03\\
\begin{center}
    \textbf{Долгополов Александр Витальевич}
    \noindent\makebox[\linewidth]{\rule{\columnwidth}{1pt}}\\
    НА ТЕМУ\\
    «Метод Зейделя»
\end{center}
\parcolumns[colwidths={1=.3\linewidth}]{2}
\colchunk{\begin{flushleft}Направление:\end{flushleft}}
\colchunk{
    \begin{flushleft}
        09.03.01 Информатика и вычислительная техника\\
    \end{flushleft}
}
\colplacechunks
\parcolumns[colwidths={1=.3\linewidth}]{2}
\colchunk{\begin{flushleft}Профиль подготовки:\end{flushleft}}
\colchunk{
    \begin{flushleft}
        «Разработка программных комплексов в рамках цифровой трансформации деятельности предприятий»
    \end{flushleft}
}
\colplacechunks
\vfill
\begin{flushleft}
    Отчёт сдан: «\rule{0.4cm}{0.2pt}» апреля 2025 г. \\
    Оценка: \rule{1cm}{0.2pt} \\
    Преподаватель \hspace{0.8cm} \rule{4cm}{0.2pt}
\end{flushleft}
\vfill
\begin{center}
    МОСКВА 2025
\end{center}
\newpage

\section*{Исходный код}
\begin{verbatim}
    #define MAX_ITER 1000

    // Условие окончания
    bool MainWindow::converge(
    const std::vector<double>& X,
    const std::vector<double>& Xp
    ) {
        double norm = 0;
        for (int i = 0; i < X.size(); i++)
            norm += (X[i] - Xp[i]) * (X[i] - Xp[i]);
        return (sqrt(norm) < ui->doubleSpinBox_epsilon->value());
    }

    // Округление
    double MainWindow::round(double x) {
        int i = 0;
        double eps = ui->doubleSpinBox_epsilon->value();
        while (eps < 1) {
            i++;
            eps *= 10;
        }
        int n = pow(10, i);
        x = int(x * n + 0.5) / double(n);
        return x;
    }


    // Поиск решения
    void MainWindow::updateResults() {
        using boost::numeric::ublas::matrix;
        QTableWidget* input_table = ui->tableWidget_input;
        QTableWidget* output_table = ui->tableWidget_output;
        QTableWidgetItem* cell;
        QString text;

        bool valid;
        double d;

        size_t row_count = input_table->rowCount();
        matrix<double> A(row_count, row_count);
        std::vector<double> B(row_count);

        // Проверка ввода
        #pragma region Input check
        for (size_t i = 0; i < row_count; i++) {
            for (size_t j = 0; j < row_count; j++) {
                cell = input_table->item(i, j);
                if (!cell || cell->text().isEmpty()) {
                    ui->label_Error->setText(
                        tr("Ошибка: ячейка %1:%2 пустая")
                        .arg(i+1).arg(j+1)
                    );
                    return;
                }
                d = cell->text().toDouble(&valid);
                if (!valid) {
                    ui->label_Error->setText(
                        tr("Ошибка: в ячейке %1:%2 указано не число")
                        .arg(i+1).arg(j+1)
                    );
                    return;
                }
                A(i, j) = d;
            }
            cell = input_table->item(i, row_count);
            if (!cell || cell->text().isEmpty()) {
                ui->label_Error->setText(
                    tr("Ошибка: ячейка %1:%2 пустая")
                    .arg(i+1).arg(row_count)
                );
                return;
            }
            d = cell->text().toDouble(&valid);
            if (!valid) {
                ui->label_Error->setText(
                    tr("Ошибка: в ячейке %1:%2 указано не число")
                    .arg(i+1).arg(row_count)
                );
                return;
            }
            B[i] = d;
        }
        ui->label_Error->setText("");
        #pragma endregion

        // Решение
        #pragma region Solution
        // X - текущая итерация
        // Xp - предыдущая итерация
        std::vector<double> X(B);
        std::vector<double> Xp;
        size_t k = 0;

        // Итерируем, пока не достигнем необходимой точности
        do {
            // Поиск решения
            Xp = X;
            for (size_t i = 0; i < row_count; i++) {
                double s = 0;
                for (size_t j = 0; j < row_count; j++)
                    if (i != j) s += A(i, j) * X[j];
                X[i] = (B[i] - s) / A(i, i);
            }
            // Проверка количества итераций
            // (выход из бесконечного цикла, если сходимость не выполняется)
            if (++k > MAX_ITER) {
                ui->label_Error->setText(
                    tr("Превышено ограничение по количеству итераций")
                );
                return;
            }
        } while (!converge(X, Xp));
        // Вывод результата
        ui->label_Error->setText(
            tr("Решение найдено за %n итераций", "", k)
        );
        for (size_t i = 0; i < row_count; i++) {
            cell = new QTableWidgetItem(locale().toString(this->round(X[i])));
            output_table->setItem(i, 0, cell);
        }
        #pragma endregion
    }
\end{verbatim}
\end{document}
